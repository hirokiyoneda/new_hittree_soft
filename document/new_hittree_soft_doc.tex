\documentclass[10.5pt]{jsarticle} 
\usepackage{graphicx}
\def\vector#1{\mbox{\boldmath $#1$}} 
\setlength{\columnsep}{3zw}
\setlength{\columnseprule}{0.4pt}
\title{new hittree soft Docment} 
\author{Hiroki Yoneda} 
\date{\today}
%%%%%%    TEXT START    %%%%%% 
\begin{document} 
\maketitle 

これまでのhittreeでは、一挙に行われていたmergeやreconstructなどを分解して解析することや、
それらのアルゴリズムをより柔軟に実装できるのを目標として、
"new hittree soft"を作成中である。
このDocumentでは、databaseや生成ファイルについて、説明する。

\section{Databese}
検出器に与えるデータベースは、detector\_map、detector\_profile, cal関数の3種類必要である。
\subsection{detector\_map}
ASICID, ASICCHとDETID, DETCHとの対応付をするデータである。
"detector\_map"というTTree形式で与えられる。
以下のようなBranchを持っている。
\begin{table}[htb]
  \begin{tabular}{|c|c|c|} \hline
  Branch名 & 型 & 説明 \\ \hline
  asicid & Int\_t & \\
  asicch & Int\_t & 使用していないchであっても記述する必要あり\\
  remapch & Int\_t & 検出器全体でのchに対する通し番号。使用していないchは、"-1"を記入。\\
  detid & Int\_t  & 検出器のID。Siを0-9に、CdTeを10以上にすることを推奨。\\
  detch & Int\_t &  検出器内でのCh。 隣接Chは、detchも隣接する。使用していないchは、"-1"を記入。\\ \hline
  \end{tabular}
\end{table}
また、asicは、64ch持っていることを前提している。

\subsection{detector\_profile}
検出器は、xy面に平行であり、各ストリップは、x軸、または、y軸に平行であることを前提としている。
検出器が3次元的に複雑に配置される場合は、hittree\_lv3から変換する必要あり。
\begin{table}[htb]
  \begin{tabular}{|c|c|c|} \hline
  Branch名 & 型 & 説明 \\ \hline
  detid & Int\_t  & 検出器のID。Siを0-9に、CdTeを10以上にすることを推奨。\\
  detch & Int\_t &  使用していないchは、ここでは記入する必要なし。\\
  detector\_material & Int\_t & 0: Si、1:CdTe\\
  detector\_HV & Int\_t & 0: Ground 1: HVside\\
  pos\_x  & Double\_t & 位置情報を持たない場合は、適当な値(0など)を入れる。\\
   pos\_y  & Double\_t & 位置情報を持たない場合は、適当な値(0など)を入れる。\\
   pos\_z  & Double\_t & \\
   delta\_x  & Double\_t & 位置情報を持たない場合は、"-1"を入れる。負値から、ストリップ方向を認識する。\\
   delta\_y  & Double\_t & 位置情報を持たない場合は、"-1"を入れる。負値から、ストリップ方向を認識する。\\
   delta\_z  & Double\_t & \\
   badch & Int\_t & 0: Good Channel 1: Bad Channel \\
   \hline
  \end{tabular}
\end{table}

\subsection{cal関数}
"calfunc\_DETID\_DETCH"という名前のTSpline3を持っているroot fileで与えられる。

\section{生成ファイル}
データの生成は、
eventtree $\to$
hitttree\_lv1 $\to$
hitttree\_lv2 $\to$
hitttree\_lv3
の3段階に分かれて行う。

hitttree\_lv1 は、データベースの適用を行うのみとする。
hitttree\_lv2 は、ストリップのマージを基本的には行う。
hitttree\_lv3 は、マージされたストリップシグナルを元に、ヒットを再構成する。
各段階で、新しいBranchが元のtreeに追加されていく形を採用する。
したがって、hittree\_lv3は、eventtree, hitttree\_lv1, hitttree\_lv2の情報も保持している。

\subsection{hitttree\_lv1}
エネルギースレッショルドは適用せず、データベースを当てるのみ。
各ブランチは、データベースの値をコピーした値になっている。
1番目以外は、全て可変長配列。
\begin{table}[htb]
  \begin{tabular}{|c|c|c|} \hline
  Branch名 & 型 & 説明 \\ \hline
 nsignal\_lv1 &  Int\_t & シグナル数\\ 
detid\_lv1[nsignal\_lv1] &  Int\_t & \\ 
detch\_lv1[nsignal\_lv1] &   Int\_t & \\ 
remapch\_lv1[nsignal\_lv1] &  Int\_t & \\ 
detector\_material\_lv1[nsignal\_lv1] & Int\_t & \\ 
detector\_HV\_lv1[nsignal\_lv1]  & Int\_t & \\
epi\_lv1[nsignal\_lv1]    & Double\_t & \\     
pos\_x\_lv1[nsignal\_lv1]   & Double\_t & \\ 
pos\_y\_lv1[nsignal\_lv1]   & Double\_t & \\ \
pos\_z\_lv1[nsignal\_lv1]   & Double\_t & \\ 
delta\_x\_lv1[nsignal\_lv1] & Double\_t & \\ 
delta\_y\_lv1[nsignal\_lv1]  & Double\_t & \\ 
delta\_z\_lv1[nsignal\_lv1]  & Double\_t & \\ \hline
  \end{tabular}
\end{table}

\subsection{hitttree\_lv2}
マージアルゴリズムを適用した後のシグナル情報。
posやdeltaは、マージアルゴリズム内で計算する必要あり。
\begin{table}[htb]
  \begin{tabular}{|c|c|c|} \hline
  Branch名 & 型 & 説明 \\ \hline
nsignal\_lv2 & Int\_t & \\
detid\_lv2[nsignal\_lv2] & Int\_t & \\
detector\_material\_lv2[nsignal\_lv2] & Int\_t & \\
detector\_HV\_lv2[nsignal\_lv2] & Int\_t & \\
n\_merged\_signal\_lv2[nsignal\_lv2] & Int\_t & 各シグナルにおいて、マージに使用した元のシグナル数\\
pos\_x\_lv2[nsignal\_lv2] & Double\_t & \\
pos\_y\_lv2[nsignal\_lv2] & Double\_t & \\
pos\_z\_lv2[nsignal\_lv2] & Double\_t & \\
delta\_x\_lv2[nsignal\_lv2] & Double\_t & \\
delta\_y\_lv2[nsignal\_lv2] & Double\_t & \\
delta\_z\_lv2[nsignal\_lv2] & Double\_t & \\
epi\_lv2[nsignal\_lv2] & Double\_t & \\
 \hline
  \end{tabular}
\end{table}

\subsection{hitttree\_lv3}
両面情報からヒット情報に変換した後のデータ。
posやdeltaは、再構成アルゴリズム内で計算する必要あり。

\begin{table}[htb]
\begin{tabular}{|c|c|c|} \hline
Branch名 & 型 & 説明 \\ \hline
nhit & Int\_t & reconstuct後のヒット数\\
detid\_lv3[nhit] & Int\_t & \\
detector\_material\_lv3[nhit] & Int\_t & \\
pos\_x\_lv3[nhit] & Double\_t & \\
pos\_y\_lv3[nhit] & Double\_t & \\
pos\_z\_lv3[nhit] & Double\_t & \\
delta\_x\_lv3[nhit] & Double\_t & \\
delta\_y\_lv3[nhit] & Double\_t & \\
delta\_z\_lv3[nhit] & Double\_t & \\
epi\_lv3[nhit] & Double\_t & \\ \hline
  \end{tabular}
\end{table}

\section{その他}
asicの最大数は、100としている。


\end{document}
